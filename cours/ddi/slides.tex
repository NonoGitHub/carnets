%! TeX program = lualatex
\documentclass[10pt]{beamer}

% \usecolortheme{seahorse}
\usetheme[numbering=fraction, block=fill, subsectionpage=progressbar]{metropolis}

\usepackage{graphicx} % allows including images
\usepackage{booktabs} % allows the use of \toprule, \midrule and \bottomrule in tables
\usepackage{natbib}         % Pour la bibliographie
\usepackage{url}            % Pour citer les adresses web
\usepackage[utf8]{inputenc} % Lui aussi
\usepackage[french]{babel} % Pour la traduction française
\usepackage{numprint}       % Histoire que les chiffres soient bien
\usepackage{amsmath}        % La base pour les maths
\usepackage{mathrsfs}       % Quelques symboles supplémentaires
\usepackage{amssymb}        % encore des symboles.
\usepackage{amsfonts}       % Des fontes, eg pour \mathbb.
\usepackage{mathtools}
\usepackage{xcolor}
\usepackage{fontspec}

\usepackage{pifont}
\usepackage{cancel}
\usepackage{hhline}
\usepackage{graphicx} % inclusion des graphiques

\usepackage{tikz}
\usepackage[framemethod=TikZ]{mdframed}

% \setbeamertemplate{footline}[page number]{}
\definecolor{myblue}{HTML}{332288}
\definecolor{bluegreen}{RGB}{3, 166, 155}
\definecolor{pitchblack}{RGB}{0, 0, 0}
\definecolor{lightbeige}{RGB}{255, 251, 241}
\definecolor{mediumgray}{RGB}{230, 230, 230}
\definecolor{darkred}{RGB}{100, 0, 0}

\hypersetup{
    colorlinks,
    citecolor=green,
    % linkcolor=darkred
}

\setbeamertemplate{navigation symbols}{}
\setbeamercolor{math text}{fg=black!15!myblue}
\setbeamercolor{frametitle}{bg=mediumgray, fg=darkred}
\setbeamercolor{alerted text}{fg=darkred}
\setbeamercovered{%
  again covered={\opaqueness<1->{15}}}
\setbeamertemplate{frametitle}[default][right]

% Ce fichier contient toutes les macros que vous pouvez avoir envie de définir
% si vous les utilisez plusieurs fois dans le document.

\PassOptionsToPackage{svgnames}{color}

% Un environnement pour bien présenter le code informatique
\newenvironment{code}{%
\begin{mdframed}[linecolor=green,innerrightmargin=30pt,innerleftmargin=30pt,
backgroundcolor=black!5,
skipabove=10pt,skipbelow=10pt,roundcorner=5pt,
splitbottomskip=6pt,splittopskip=12pt]
}{%
\end{mdframed}
}


\newcommand\N{\mathbb{N}}
\newcommand\R{\mathbb{R}}
\newcommand\E{\mathbb{E}}
\newcommand\Pb{\mathbb{P}}
\newcommand\va[1]{\mathbf{#1}}


\newcommand{\argmin}{\mathop{\arg\min}}                     % Arg-min
\newcommand{\st}{\text{s.t.}}                            % ``s.t.''

%%%%% Math\'{e}matiques g\'{e}n\'{e}rales
\newcommand{\defegal}{:=}                                   % D\'{e}finition

\newcommand{\bgfset}[2]{\big\{#1\:\big|\:#2\big.\big\}}     % Ensemble
\newcommand{\Bgfset}[2]{\Big\{#1\:\Big|\:#2\Big.\Big\}}
\newcommand{\defset}[2]{\left\{#1\:\left|\:#2\right.\right\}}

\newcommand{\IFF}{iff~}                                     % if and only if

\newcommand{\bbN}{\mathbb{N}}                               % Entiers naturels
\newcommand{\bbR}{\mathbb{R}}                               % Nombres r\'{e}els
\newcommand{\bbRb}{\overline{\mathbb{R}}}                   % Completion de R

\newcommand{\abs}[1]{\left|#1\right|}                       % Valeur absolue
\newcommand{\norm}[1]{\left\|#1\right\|}                    % Norme
\newcommand{\sqnorm}[1]{\left\|#1\right\|^{2}}              % Norme au carr\'{e}

\newcommand{\dd}{\,\mathrm{d}}                              % d de derivation
\newcommand{\dpp}[2]{\dd #1 (#2)}                           % dP(omega)

\newcommand{\derpar}[2]{\frac{\partial#1}{\partial#2}}      % D\'{e}riv\'{e}e partielle
\newcommand{\dertot}[2]{\frac{\mathrm{d}#1}{\mathrm{d}#2}}  % D\'{e}riv\'{e}e totale

\newcommand{\projop}[1]{\mathrm{proj}_{#1}}                 % Op\'{e}rateur projection
\newcommand{\proj}[2]{\projop{#1}\left(#2\right)}           % Projection
\newcommand{\dist}[2]{\mathrm{dist}_{#1}\left(#2\right)}    % Distance

%%%%% Parenth\`{e}ses, crochets et accolades

\newcommand{\np}[1]{(#1)}                                   % Parenth\`{e}se normal
\newcommand{\bp}[1]{\big(#1\big)}                           % Parenth\`{e}se big
\newcommand{\Bp}[1]{\Big(#1\Big)}                           % Parenth\`{e}se Big
\newcommand{\bgp}[1]{\bigg(#1\bigg)}                        % Parenth\`{e}se bigg
\newcommand{\Bgp}[1]{\Bigg(#1\Bigg)}                        % Parenth\`{e}se Bigg

\newcommand{\nc}[1]{[#1]}                                   % Crochet normal
\newcommand{\bc}[1]{\big[#1\big]}                           % Crochet big
\newcommand{\Bc}[1]{\Big[#1\Big]}                           % Crochet Big
\newcommand{\bgc}[1]{\bigg[#1\bigg]}                        % Crochet bigg
\newcommand{\Bgc}[1]{\Bigg[#1\Bigg]}                        % Crochet Bigg

\newcommand{\na}[1]{\{#1\}}                                 % Accolade normal
\newcommand{\ba}[1]{\big\{#1\big\}}                         % Accolade big
\newcommand{\Ba}[1]{\Big\{#1\Big\}}                         % Accolade Big
\newcommand{\bga}[1]{\bigg\{#1\bigg\}}                      % Accolade bigg
\newcommand{\Bga}[1]{\Bigg\{#1\Bigg\}}                      % Accolade Bigg

%%%%% Versions avec s\'{e}parateur

\newcommand{\nps}[2]{\np{#1\mid#2}}                         % Parenth\`{e}se normal
\newcommand{\bps}[2]{\bp{#1\ \big|\ #2}}                    % Parenth\`{e}se big
\newcommand{\Bps}[2]{\Bp{#1\ \Big|\ #2}}                    % Parenth\`{e}se Big
\newcommand{\bgps}[2]{\bgp{#1\ \bigg|\ #2}}                 % Parenth\`{e}se bigg
\newcommand{\Bgps}[2]{\Bgp{#1\ \Bigg|\ #2}}                 % Parenth\`{e}se Bigg

\newcommand{\ncs}[2]{\nc{#1\mid#2}}                         % Crochet normal
\newcommand{\bcs}[2]{\bc{#1\ \big|\ #2}}                    % Crochet big
\newcommand{\Bcs}[2]{\Bc{#1\ \Big|\ #2}}                    % Crochet Big
\newcommand{\bgcs}[2]{\bgc{#1\ \bigg|\ #2}}                 % Crochet bigg
\newcommand{\Bgcs}[2]{\Bgc{#1\ \Bigg|\ #2}}                 % Crochet Bigg

\newcommand{\nas}[2]{\na{#1\mid#2}}                         % Accolade normal
\newcommand{\bas}[2]{\ba{#1\ \big|\ #2}}                    % Accolade big
\newcommand{\Bas}[2]{\Ba{#1\ \Big|\ #2}}                    % Accolade Big
\newcommand{\bgas}[2]{\bga{#1\ \bigg|\ #2}}                 % Accolade bigg
\newcommand{\Bgas}[2]{\Bga{#1\ \Bigg|\ #2}}                 % Accolade Bigg

%%%%% Raccourcis package maths
\newcommand{\cA}{\mathcal{A}}
\newcommand{\cB}{\mathcal{B}}
\newcommand{\cC}{\mathcal{C}}
\newcommand{\cD}{\mathcal{D}}
\newcommand{\cE}{\mathcal{E}}
\newcommand{\cF}{\mathcal{F}}
\newcommand{\cG}{\mathcal{G}}
\newcommand{\cH}{\mathcal{H}}
\newcommand{\cI}{\mathcal{I}}
\newcommand{\cJ}{\mathcal{J}}
\newcommand{\cK}{\mathcal{K}}
\newcommand{\cL}{\mathcal{L}}
\newcommand{\cM}{\mathcal{M}}
\newcommand{\cN}{\mathcal{N}}
\newcommand{\cO}{\mathcal{O}}
\newcommand{\cP}{\mathcal{P}}
\newcommand{\cQ}{\mathcal{Q}}
\newcommand{\cR}{\mathcal{R}}
\newcommand{\cS}{\mathcal{S}}
\newcommand{\cT}{\mathcal{T}}
\newcommand{\cU}{\mathcal{U}}
\newcommand{\cV}{\mathcal{V}}
\newcommand{\cW}{\mathcal{W}}
\newcommand{\cX}{\mathcal{X}}
\newcommand{\cY}{\mathcal{Y}}
\newcommand{\cZ}{\mathcal{Z}}

\newcommand{\BB}{\mathbb{B}}
\newcommand{\CC}{\mathbb{C}}
\newcommand{\DD}{\mathbb{D}}
\newcommand{\EE}{\mathbb{E}}
\newcommand{\FF}{\mathbb{F}}
\newcommand{\GG}{\mathbb{G}}
\newcommand{\HH}{\mathbb{H}}
\newcommand{\II}{\mathbb{I}}
\newcommand{\JJ}{\mathbb{J}}
\newcommand{\KK}{\mathbb{K}}
\newcommand{\LL}{\mathbb{L}}
\newcommand{\MM}{\mathbb{M}}
\newcommand{\NN}{\mathbb{N}}
\newcommand{\OO}{\mathbb{O}}
\newcommand{\PP}{\mathbb{P}}
\newcommand{\QQ}{\mathbb{Q}}
\newcommand{\RR}{\mathbb{R}}
\newcommand{\TT}{\mathbb{T}}
\newcommand{\UU}{\mathbb{U}}
\newcommand{\VV}{\mathbb{V}}
\newcommand{\WW}{\mathbb{W}}
\newcommand{\XX}{\mathbb{X}}
\newcommand{\YY}{\mathbb{Y}}
\newcommand{\ZZ}{\mathbb{Z}}



\title{Décision dans l'incertain}
\subtitle{Modélisation d'une épidémie}
\date{Vendredi 29 Mai 2020}

\begin{document}

\begin{frame}
  \maketitle
\end{frame}

\begin{frame}{Comment modéliser une épidémie ?}
  \begin{columns}
    \column{.5\textwidth}
    \includegraphics[width=.9\textwidth]{cubic_model}
    \includegraphics[width=.9\textwidth]{article}
    \column{.5\textwidth}
    \begin{itemize}
      \item Beaucoup de modèles sont proposés ...
      \item ... avec des enjeux énormes
    \end{itemize}
    \emph{Comment juger de la pertinence d'un modèle ?}

    \vspace{.5cm}

    \pause
    \alert{Nous verrons ici comment utiliser \\
    un modèle \emph{probabiliste} pour modéliser
    une épidémie}

    \vspace{.5cm}

    \emph{Le travail présenté ci-après est adapté très largement
      adapté de Cosma Shalizi dans sa présentation \emph{Epidemic Models}
      (16 avril 2020)
    }
  \end{columns}
\end{frame}

\section{D'une modélisation probabiliste au modèle SIR}

\begin{frame}{Le modèle SIR part des présupposés suivants}
  Supposons que nous ayons une maladie avec 3 états possibles
    \begin{itemize}
      \item \emph{S (Susceptible)}: sain, mais peut être contaminé
      \item \emph{I (Infectious)}: malade, et peut contaminer d'autres personnes
      \item \emph{R (Removed)}: guéri, et ne peut plus être contaminé
    \end{itemize}

  \vspace{.5cm}
  \begin{block}{Jeu d'hypothèses}
    \begin{itemize}
      \item \textbf{Contagion :} un $S$ peut être contaminé quand il rencontre un $I$
      \item \textbf{Guérison} au bout d'un certain temps, les $I$ se transforment spontanément en $R$
      \item \textbf{Mixité:} la \emph{probabilité} qu'un $S$ rencontre un $I$ dépend du nombre total
        de $S$, $I$ et $R$
    \end{itemize}
  \end{block}
\end{frame}

\begin{frame}{Un premier modèle probabiliste}
  Proposons un premier modèle \emph{discret}, suivant les hypothèses précédentes
  \begin{itemize}
    \item Nous avons une population totale de $n$
    \item Le pas de temps de discrétisation $\delta$ est petit \\
    \item On note $(S_k, I_k, R_k)$ l'état de l'épidémie au pas de temps $t_k = t_0 + k\delta$
      \begin{itemize}
        \item Pendant un temps $\delta$, chaque personne $S$ rencontre $p\delta$ personnes,\\
          avec une probabilité de contagion de $c$
        \item Nous avons une probabilité $\frac{I_k}{n}$ que chaque personne rencontrée \\ soit infectée
        \item la probabilité qu'une personne $S$ soit contaminée durant l'intervalle de temps est dès-lors
          \[
            \begin{aligned}
              \PP(S \to I) &= 1 - \PP(S \to S)  \\
                           &= 1 - \underbrace{(1 - c \frac{I_k}{n})^{p \delta}}_{\mathclap{\text{(aucune rencontre ne mène à une contamination)}}} \\
              \PP(S \to I)&\approx p \delta \times c \times \frac{I_k}{n}
            \end{aligned}
          \]
      \end{itemize}
  \end{itemize}
\end{frame}

\begin{frame}{Contagions, guérisons}
  De ce qui précède, nous déduisons
  \begin{block}{Propriété}
    \begin{itemize}
      \item Le nombre de contagion $C_k$ ($S \to I$) suit une \emph{loi binomiale}
        \[
          C_k \sim \text{binom}(S_k, pc\delta \frac{I_k}{n})
        \]
        \vspace{-.5cm}
      \item De même, on en déduit que le nombre de rémission $D_k$ ($I \to R$) \\
        suit une loi binomiale
        \[
          D_k \sim \text{binom}(I_k, \gamma \delta)
        \]
        avec $\gamma$ taux de guérison
    \end{itemize}
  \end{block}
\end{frame}

\begin{frame}{Nous obtenons les équations probabilistes suivantes ...}
  \[
    \left\{
      \begin{aligned}
        % C_{k} &\sim \text{binom}(S_k, \frac{\beta}{n}\delta I) \\
        % D_{k} &\sim \text{binom}(I_k, \gamma \delta) \\
        S_{k+1} &= S_k - C_k \\
        I_{k+1} &= I_k + C_k - D_k \\
        R_{k+1} &= R_k + D_k
      \end{aligned}
    \right.
  \]

  \vspace{1cm}

  avec
  \begin{itemize}
    \item $C_k\sim \text{binom}(S_k, \frac{\beta}{n}\delta I_k)$: nombre de contagions au pas de temps $k$
    \item $D_k\sim \text{binom}(I_k, \gamma \delta)$: nombre de rémissions au pas de temps $k$
    \item $\beta = p c$ taux de contagion
  \end{itemize}

  \vspace{.5cm}
  \pause
  Notons que le passage entre les trois états se fait uniquement \\
  dans le sens $S \to I \to R$!
\end{frame}

\begin{frame}{Simulation (1)}
  Si nous faisons tourner une simulation
  \begin{figure}
    \includegraphics[width=.9\textwidth]{simu_1}
  \end{figure}
\end{frame}

\begin{frame}{Simulation (2)}
  N'oublions pas que nous sommes dans une configuration aléatoire ! \\
  En faisant tourner 100 simulations, nous avons
  \begin{figure}
    \includegraphics[width=.9\textwidth]{simu_2}
  \end{figure}
\end{frame}

\begin{frame}{Simulation (3)}
  Si la contagion devient plus difficile
  \begin{figure}
    \includegraphics[width=.9\textwidth]{simu_3}
  \end{figure}
\end{frame}

\begin{frame}{Simulation (4)}
  Niveau de contagion intermédiaire
  \begin{figure}
    \includegraphics[width=.9\textwidth]{simu_4}
  \end{figure}
\end{frame}

\begin{frame}{Simulation (5)}
  Probabilité de guérison plus élevée
  \begin{figure}
    \includegraphics[width=.9\textwidth]{simu_5}
  \end{figure}
\end{frame}

\begin{frame}{Et si nous passions à la limite ?}
  Que se passe-t-il si la population devient très grande ($n \to \infty$) \\
  et le pas de temps très petit ($\delta \to 0$) ?
  \begin{itemize}
    \item On a:
    \[
      \begin{aligned}
        \dfrac{S_{k+1} - S_k}{\delta} &\sim - \frac{1}{\delta} \text{binom}(S_k, \beta \delta I_k / n)
      \\
        \dfrac{I_{k+1} - I_k}{\delta} &\sim \frac{1}{\delta} \text{binom}(S_k, \beta \delta I_k / n) -
        \frac{1}{\delta} \text{binom}(I_k, \gamma \delta)
      \end{aligned}
    \]
  \item En passant à l'espérance (si $X \sim \text{binom}(n, p)$, $\EE(X) = n p$)
    \[
      \begin{aligned}
        \EE\bc{\dfrac{S_{k+1} - S_k}{\delta}} &= -\frac{\beta}{n} S_k I_k
      \\
        \EE\bc{\dfrac{I_{k+1} - I_k}{\delta}} &= \frac{\beta}{n} S_k I_k - \gamma I_k
      \end{aligned}
    \]
  \end{itemize}
\end{frame}

\begin{frame}{Qu'en est-il des fluctuations autour de la moyenne ?}
  \begin{itemize}
    \item Si on regarde la variance (si $X \sim \text{binom}(n, p)$, $\VV(X) = n p (1- p)$)
    \[
        \VV\bc{\dfrac{S_{k+1} - S_k}{\delta}} = \frac{\beta}{\delta n} S_k I_k (1 - \frac{\beta}{n} \delta I_k)
    \]
    et on obtient $\VV\bc{\dfrac{S_{k+1} - S_k}{\delta}} \to 0$ quand $n \to \infty$
  \item Le même raisonnement tient pour $\VV\bc{\dfrac{I_{k+1} - I_k}{\delta}}$
  \item On peut donc négliger les variations autour de la moyenne quand $n \to \infty$!
  \end{itemize}

  \emph{Ce passage n'est pas trop rigoureux...}
\end{frame}

\begin{frame}{Le modèle SIR}
  Lorsque $n\to \infty$ et $\delta \to 0$
  on obtient le modèle SIR continu :

  \begin{block}{Modèle SIR}
    Le modèle SIR correspond aux équations différentielles suivantes :
    \[
      \left\{
        \begin{aligned}
          & \dot{S} = - \frac{\beta}{n} S I \\
          & \dot{I} = \frac{\beta}{n} I S - \gamma I \\
          & \dot{R} = \gamma I
        \end{aligned}
      \right.
    \]
  \end{block}
  Ces équations sont déterministes et non-linéaires !
\end{frame}

\begin{frame}{Simulation avec les paramètres originaux}
  On résout le système d'équation différentielle numériquement
  \begin{figure}
    \includegraphics[width=.9\textwidth]{simu_6}
  \end{figure}
\end{frame}

\begin{frame}{Qu'est-ce que $R_0$?}
  En normalisant les équations du modèle SIR, nous observons que le seul
  paramètre restant est le ratio $\frac{\beta}{\gamma}$.

  \begin{block}{Nombre de reproduction basique $R_0$}
   On définit $R_0$ comme le nombre moyen de nouvelles infections si on
   ajoute un agent infectieux dans une population non atteinte
  \end{block}

  Nous avons trois régimes :
  \begin{itemize}
    \item $R_0 < 1$: Sous-critique
    \item $R_0 > 1$: Super-critique
    \item $R_0 = 1$: Critique
  \end{itemize}
\end{frame}

\begin{frame}{Identification de $R_0$}
  Dans le modèle SIR, on identifie :
  \[
    R_0 = \frac{\beta}{\gamma}
  \]
  On suppose qu'initialement: $S(0) \approx n$. Alors
  \[
    \begin{aligned}
      \dot{I} &= \frac{\beta}{n} S I - \gamma I  \\
              &= (\beta \frac{S}{n} - \gamma) I \\
              &\approx (\beta  - \gamma) I \\
    \end{aligned}
  \]
  d'où une croissance exponentielle de malades au début de l'épidémie
  \[
  I(t) \approx I_0 e^{(\beta - \gamma)t}
  \]
\end{frame}

\begin{frame}{Les limites du modèle SIR}
  Le modèle SIR comporte seulement trois états ! Des variantes existent :
  \begin{itemize}
    \item Etat d'incubation $E$ des gens qui ont été exposés à la maladie mais ne sont pas
      encore dans l'état $I$ (modèle SEIR)
    \item Etat asymptomatique $A$ des gens ne présentant pas de symptôme
      mais contaminant
    \item Possibilité éventuelle de retomber malade (passage $R \to I$)
  \end{itemize}

  Le modèle dépend donc intrinséquement des caractéristiques du virus ...
\end{frame}

\begin{frame}{Un modèle régionalisé}
  Une extension usuelle du modèle SIR est de prendre en compte les échanges
  entre plusieurs régions $i = 1, \cdots, N$
  \[
    \dot{I_i} = \frac{\beta}{n} S_i I_i - \gamma I_i - \mu_{ij} I_i +
    \mu_{ji} I_j
  \]
  où $\mu_{ij}$ est la proportion de personnes voyageant de la région $i$ à $j$

  \vspace{1cm}

  \pause
  \emph{
  Pendant l'épidémie, $\mu_{ij}$ a très souvent été estimé avec les données
  téléphoniques
  }
\end{frame}

\section{Un modèle plus détaillé}

\begin{frame}{}
  \begin{columns}
    \column{.5\textwidth}
    \includegraphics[width=\textwidth]{social_network}
    {\tiny
      Source: Wikipedia
    }
    \column{.5\textwidth}
    Nous avons vu précédemment le modèle SIR,\\
    qui reposait sur un certain nombre d'hypothèses\\[1cm]

    Analysons maintenant comment pouvons-nous adapter ce modèle sur un graphe
  \end{columns}
\end{frame}

\begin{frame}{Structure en réseau}

  \begin{itemize}
    \item Certaines personnes sont plus sociables que d'autres ...
    \item Intéressons nous maintenant au \emph{graphe des relations} \\
      (si $i$ (Igor) connaît $j$ (Jeanne), alors $a_{ij} = 1$, sinon $0$)
  \end{itemize}

  \begin{block}{Définitions (rappel)}
    \begin{itemize}
      \item Le \emph{degré} d'un noeud $i$ est égal à
        \[
          \text{deg}(i) = \sum_{j=1}^n a_{ij}
        \]
        \vspace{-.4cm}
      \item Pour tout $k \in \NN$, on note $p(k)$ la fraction des noeuds ayant\\
        un degré égal à $k$
    \end{itemize}
  \end{block}

  Comment la topologie du réseau influe-t-elle sur la diffusion de l'épidémie ?
\end{frame}

\begin{frame}{Passons au modèle mathématique}
  \begin{itemize}
    \item Notons $\tau$ le taux de transmission de la maladie
    \item Supposons que $i$ soit infecté par un noeud adjacent
    \item $R_0$ est le nombre moyen de nouvelles infections causées par $i$,
      soit
      \[
        R_0 = \tau \EE\bc{\text{deg}(i) - 1}
      \]
      (on enlève le noeud adjacent qui a transmis la maladie à Igor)
  \end{itemize}

  Calculons maintenant $\EE\bc{\text{deg}(i) - 1}$ pour avoir $R_0$
\end{frame}

\begin{frame}{Le paradoxe de l'amitié}
  Attention !
  \[
    \EE\bc{\text{deg}(Igor)} \neq \EE\bc{\text{degree of random node}} = \sum_{k=0}^\infty k p(k)
  \]

  \begin{block}{Le paradoxe de l'amitié}
    Mes amis ont en moyenne plus d'amis que moi
  \end{block}
  \textbf{Preuve. }\\
  Igor \emph{n'est pas} un noeud aléatoire, mais un noeud atteint par une \emph{arête aléatoire} \\

  En moyenne, les noeuds avec une connectivité élevée ont une probabilité plus grande
  d'être atteint par une arête prise de manière aléatoire \\

  \[
    \PP\bp{\text{deg}(Igor) = k} = \dfrac{k p(k)}{\sum_{k=1}^\infty k p(k)}
  \]
  puis nous prouvons le paradoxe en prenant l'espèrance de cette loi
\end{frame}

\begin{frame}{Retournons à notre $R_0$}
  On a alors
  \[
    \begin{aligned}
      R_0 &= \tau \EE\bc{\text{deg}(i) - 1} \\
          &= \tau \dfrac{\sum_{k=1}^\infty (k-1) k p(k)}{\sum_{k=1}^\infty k p(k)} \\
          &= \tau \dfrac{\EE\bc{K^2 - K}}{\EE\bc{K}}
    \end{aligned}
  \]
  \vspace{.7cm}
  D'où
  \begin{block}{}
  \[
    R_0 = \tau \Bp{\dfrac{\VV\nc{K} + \EE\nc{K}^2 - \EE\nc{K}}{\EE\nc{K}}}
  \]
  \end{block}
\end{frame}

\begin{frame}{Comment éviter l'épidémie ? }
  On souhaite avoir $R_0 < 1$, soit
  \[
    \tau \Bp{\dfrac{\VV\bc{K}}{\EE\bc{K}} + \EE\bc{K} - 1} < 1
  \]
  Pour cela, on peut :
  \begin{itemize}
    \item Diminuer la transmission $\tau$ \emph{(sortir masqué)}
    \item Diminuer le degré moyen $\EE\nc{K}$ \emph{(rester chez soi)}
    \item Diminuer la variance $\VV\nc{K}$ \emph{(enlever les noeuds très connectés)}
  \end{itemize}
\end{frame}

\begin{frame}{Comment simuler l'épidémie sur un réseau ?}
  \begin{itemize}
    \item En pratique, il est très difficile d'avoir accès à des graphes
      réalistes (hormi pour Facebook et quelques opérateurs téléphoniques)
    \item On se contente souvent de \emph{graphes aléatoires} pour les simulations !
      Par exemple, en utilisant des graphes d'Erdös-Rényi
  \end{itemize}

  \begin{block}{Graphe d'Erdös-Rényi}
    Soit $n\geq 1$ un entier, $p \in [0, 1]$ un paramètre\\
    Les arêtes graphe aléatoire d'Erdös-Rényi $G = G(n, p)$ sont générées
    de manière aléatoire et indépendante
    \[
      a_{ij} = 1 \quad \iff \quad X_{ij} = 1
    \]
    où $X_{ij}$ est une variable aléatoire suivant une loi de Bernouilli de paramètre $p$
  \end{block}
\end{frame}

\begin{frame}{Vers les modèles agents}

  {\footnotesize
    \begin{columns}
      \column{.3\textwidth}
      \includegraphics[width=.9\textwidth]{conway}
      \column{.7\textwidth}
      \begin{itemize}
        \item Un modèle agent est une simulation avec $n$ agents où
          chacun des agents suit et interagit avec les autres agents suivant
          un comportement programmé
        \item Des comportements systèmiques peuvent parfois émerger des interactions
          individuelles
        \item Le \emph{game of life} de Conway est un des modèles agents les plus connus,
          aboutissants suivant l'initialisation à des comportements complexes des
          automates cellulaires
      \end{itemize}

    \end{columns}

    \vspace{1cm}
    Pour aller plus loin, voir la vidéo très intéressante de Guy Théraulaz : \emph{L'intelligence collective des sociétés animales}
    \footnote{
      \url{https://www.college-de-france.fr/site/colloque-2018/symposium-2018-10-19-11h50.htm}
    }
  }
\end{frame}

\end{document}
